\chapter{Conclusion}\label{ch:conclusion}

A plethora of research has already been conducted into general ranking algorithms for SNS feeds. We successfully aggregated and implemented common algorithms proposed in the literature by Aga~\cite{Aga2014}, Li~\cite{LiTiaLee2010} and Szo~\cite{szomszor2008semantic} as a basis to our ranking algorithm and on top of that incorporated the results of our user modelling to enable a more personal factor to the algorithm.

As discussed in the evaluation section, it proved to be quite difficult for users to determine what type of posts they prefer, let alone generalise their preferences to certain user types. This makes it very difficult for us to confirm how effective our algorithm is exactly. Some suggestions in regard to this are proposed in the future work section.

\section{Future Work}
 
Our algorithm only contains a few user types, these user types could be refined into more specific categories thus accommodating a larger variety of users, but doing so would require more justification via surveying. There could also be more work dedicated to making the user understand each user type, for example providing examples of typical posts that may be scored highly for each user type. This will allow the user to more easily and correctly select the user type that best defines themselves. 

And most importantly, the solution that this thesis proposes would best be used as a framework to implement similar ranking algorithms across other SNS and further, more extensive and large-scale evaluation should be performed if such an algorithm was to be seriously considered for use within these SNS.